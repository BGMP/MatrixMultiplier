\documentclass[12pt]{report}
\usepackage{amsfonts}
\usepackage{amsmath}
\usepackage{caption}
\usepackage{geometry}
\usepackage{graphicx}
\usepackage{hyphenat}
\usepackage{pgfplots}
\usepackage[spanish]{cleveref}
\usepackage[spanish, es-nodecimaldot, es-tabla]{babel}
\pgfplotsset{compat=1.18}

\graphicspath{{img/}}

\newgeometry{
  right=3cm,
  left=3cm,
  top=2.5cm,
  bottom=2.5cm
}

\newenvironment{longlisting}{\captionsetup{type=listing}}{}

\begin{document}
  \begin{center}
    \section*{Tarea 1}
    
    Algoritmos y Estructuras de Datos Avanzados / Magister en Cs. de la Computación
  \end{center}
  
  \textbf{Respuestas:}
  
  \begin{enumerate}
    \item Usando las ideas anteriores, generar al azar las
    matrices A y B (considere matrices de enteros) y
    completar las siguiente tabla con los tiempos de
    ejecución1. DR1 usa la primera propiedad, y DR2
    usa la segunda (prográmelos en el lenguaje que
    estime conveniente).
  \end{enumerate}
  
  Para la resolución del punto 1, se utilizó el lenguaje de programación \textbf{Python}, con el cual se implementaron el algoritmo tradicional, y los algoritmos identificados como DR1 y DR2 dentro del enunciado.
  
  Esta implementación se trata de un programa que permite al usuario definir un número entero ''\textit{n}'', con el cual genera, de manera automática, dos matrices cuadradas de n × n con valores aleatorios. Luego, estas dos matrices son multiplicadas a través de los tres algoritmos ya mencionados. Finalmente, el tiempo de ejecución de cada uno de ellos es mostrado por pantalla para poder compararlos en la tabla a continuación.
  
  \begin{center}
    \begin{tabular}{ | c | p{5.5cm} | p{3.5cm} | p{3.5cm} |}
      \hline
      {} & \multicolumn{3}{|c|}{\textbf{Tiempos}} \\
      \hline
      \textbf{n} & {\textbf{Algoritmo Tradicional}} & {\textbf{DR1}} & {\textbf{DR2}}\\ \hline
      {\textbf{32}} & 30 ms & 0 ms & 0 ms \\ \hline
      {\textbf{64}} & 242 ms & 0 ms & 0 ms \\ \hline
      {\textbf{128}} & 1925 ms & 2 ms & 1 ms \\ \hline
      {\textbf{256}} & 15347 ms & 21 ms & 18 ms \\ \hline
      {\textbf{512}} & 124653 ms & 267 ms & 165 ms \\ \hline
      {\textbf{1024}} & 1002550 ms & 6928 ms & 3002 ms \\ \hline
      {\textbf{2048}} & N/A & N/A & N/A \\ \hline
      {\textbf{4060}} & N/A & N/A & N/A \\ \hline
    \end{tabular}
  \end{center}
  
  Para efectos de esta tarea, solo se han considerado valores para $ n \leq 1024 $, ya que los tiempos de espera son razonables dentro de ese rango.
  
  \begin{enumerate}
    \setcounter{enumi}{1}
    \item Obtenga al menos dos conclusiones, respecto del rendimiento de los algoritmos.
  \end{enumerate}
  
  \begin{itemize}
    \item \textbf{Conclusión 1}: A medida que el orden de las matrices (\textit{n}) aumenta, el rendimiento del algoritmo tradicional se vuelve notablemente peor que el de los algoritmos DR1 y DR2.
    \item \textbf{Conclusión 2}: Los algoritmos DR1 y DR2 escalan mucho mejor que el algoritmo tradicional, ya que sus tiempos de ejecución crecen mucho más lento a medida que aumenta el orden de las matrices (\textit{n}). Gracias a esto, DR1 y DR2 son mucho mejores para trabajar con matrices de orden mayor.
  \end{itemize}
  
  \newpage
  
  \begin{enumerate}
    \setcounter{enumi}{2}
    \item Haga un estudio de comportamiento asintótico de los 2 algoritmos que creó.
  \end{enumerate}

  \centering
    \makebox[0pt]{\includegraphics[width=15cm]{TRADITIONAL_PLOT.png}}
  
  \centering
    \makebox[0pt]{\includegraphics[width=15cm]{DR1_PLOT.png}}
  
  \centering
    \makebox[0pt]{\includegraphics[width=15cm]{DR2_PLOT.png}}

  \centering
    \makebox[0pt]{\includegraphics[width=15cm]{DR1_DR2_PLOT.png}}
    
  \centering
    \makebox[0pt]{\includegraphics[width=15cm]{TRADITIONAL_DR1_DR2_PLOT.png}}
\end{document}


